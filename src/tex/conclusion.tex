\chapter{Conclusion}

In these chapters 4 basic underlying techniques (\autoref{ch:basic underlying techniques}) and a technique for every major category (\autoref{ch:feature location techniques}) are presented, which are only the peak of the iceberg. Given an example of the Freemind Mind mapping software (\autoref{ch:Freemind Example}) and explaining the techniques based on it the general function of feature location techniques are shown. Overall the conclusion of the techniques is an obvious, but not irrelevant statement: \newline
\begin{center}
	No \textit{Feature Location Technique} can be perfect in every kind of scenario, but by knowing the exact fitting assumptions the best resulting technique can be chosen.
\end{center} 

Feature Location Techniques are an essential part of software development and in preparing the steps towards Software Product Line Engineering. Reality shows that even with the best architecture and planning software project tend to be a complex and highly branched construct, which are almost impossible to be analysed without any help of techniques. \newline
The field of Feature Location is not a new upcoming issue, but never the less it is still up-to-date. The growth of software projects keeps increasing and therefore the amount of issues that comes along. Common techniques are still refined and new techniques are developed in a constant manner. New heuristics of weighting are neglecting aspects can change a technique from an average resulting technique to one that has the ability to locate features like no other. But the even best technique is only reliable on a decent fit to its assumptions. \newline



\cleardoublepage
