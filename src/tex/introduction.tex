\chapter{Introduction}
%\section{Eins - Eins}
%\subsection{Eins - Eins - Eins}

% Die Logos sind veraltet und duerfen zurzeit nicht verwendet werden!
% Auf Seite \pageref{Logo} in Abbildung \ref{Logo} befindet sich das SE Logo.

A feature location technique is aiming at the locating of software artifacts as a realization of a system requirement. It could be \emph{functional}, like the ability of doing a special kind of computation for example counting elements like a Log-In, or it could be \emph{non-functional}, i.e. completing the Log-In within 5 seconds.
To be able to understand the approach strategy of a feature location technique and to derive a measurement of the result it is necessary to have a basic knowledge about two aspects of modern software engineering. Without either one of the following two underling definitions it's is not clearly definable what a feature location technique should be capable of and there is also no way to rate if a technique is efficient and correct.

On the one hand there are the features. As defined by the Institute of Electrical and Electronics Engineers (IEEE) a feature is defined as 'A distinguishing characteristic of a software item (e.g., performance, portability, or functionality)' \cite{wiki:Softwarefeature}. Simplified a feature is a software artifact implementing a given requirement. Features are often described by the definition of \emph{Rajlich and Chen}, who describe a feature or concept as a triple of \textit{name}, the name of the feature, \textit{intension}, a short precise description, and \textit{extension}, the artifacts implementing the feature \cite{KR00} \label{Rajlich_Chen}.

The other hand there is the software product line engineering (SPLE). A product line is a variety of products, which in our case are software products, which 
\begin{quote}
	"share a common, managed set of features satisfying the specific needs a particular market segment or mission and that are developed from a common set of core assets in a prescribed way." \cite{SPL}
\end{quote}
A good example are the products of  SAP like the \emph{Business One}, \emph{Business All-In-One} and \emph{Business ByDesign}, which share a basic set of functionality, build up on each other and in most cases are modified to fit the exact needs of a customer.
The SPLE promotes \textit{systematic}  software reuse being base on the knowledge about the set of available features, relationships among the features and the relationship between features and their artifacts.
The most essential step for unfolding the complexity of existing implementations to be able to transform it into a SPLE includes the identification of the implemented features and their corresponding artifacts.

The locating and defining of a feature is the problem a feature location technique should solve, so that developers of software product lines are supported during the maintenance and the aspect-/feature oriented refactoring of software. 
%\cleardoublepage
