\vspace*{2cm}
% Abstract
%{\bf\Large Kurzfassung} \\ [1em] 
%Eine kurze Zusammenfassung der Arbeit.

%\vspace{10ex}
{\bf\Large Abstract} \\ [1em]
Locating software artifacts that implement a specific program functionality, whether it's functional or non-functional, are called a feature. Detecting features in a program is the main goal of Feature Location Techniques(FLT). It assists software developers during the maintenance and refactoring of the code.
But also the software product line engineering(SPLE), which specifies, designed and implements different products by managing features, uses these techniques to create a product without copying code unstructured but by systematic reuse of the artifacts the FLT's locate \cite{pohl2005software}.

Therefore my seminar paper deals with different feature location techniques from very fundamental methods to some of today's newest research fields. In this paper I introduce a real use case example, to show the real utility of the techniques, of the Freemind mind mapping software \cite{FrM16}.

In this paper we continue to get to know to the basics of FLT's to understand how they are able to define artifacts, the classification of FLT's considering their approach strategy, explaining different techniques of different previously mentioned classes, regarding their strengths and weaknesses, on a realistic use case of a real software segment.
At the end will be an outlook to leveraging SPLE architectures and possible improvements of the existing techniques\cite{zhao2006sniafl}.

\cleardoublepage
